\documentclass{cv}

\usepackage[left=0.5in,top=0.5in,right=0.5in,bottom=0.5in]{geometry}    % Document margins
\usepackage[fixed]{fontawesome5}
% \usepackage[default]{sourcesanspro}
% \usepackage[T1]{fontenc}

\newcommand{\tab}[1]{\hspace{.2667\textwidth}\rlap{#1}}
\newcommand{\itab}[1]{\hspace{0em}\rlap{#1}}

\sidebar{header.png}
\imagename{face.png}
\name{K. D. Sunera Avinash Chandrasiri}                                 % Name
\subtitle{344/1, Moonamalgahawatta, Duwa Temple Road, Kalutara South.}  % Address
\subtitle{(076) 833 6850 \\ \href{mailto:suneraavinash.17@cse.mrt.ac.lk}{suneraavinash.17@cse.mrt.ac.lk}} 
\subtitle{\url{https://kdsuneraavinash.me}} 
\header{true}
% Phone number, email

\begin{document}

%----------------------------------------------------------------------------------------
%	EDUCATION SECTION
%----------------------------------------------------------------------------------------

\begin{rSection}{PROFILE}
    A third-year computer science undergraduate who has a passion for algorithms and problem-solving.
    I am also interested in web technologies and mobile app development.
    A GNU/Linux Enthusiast. \\
    \faGlobe\ Portfolio: \url{https://kdsuneraavinash.me} \\
    \faGithub\ Github profile: \url{https://github.com/kdsuneraavinash} \\
    \faLinkedin\ LinkedIn profile: \url{https://www.linkedin.com/in/kdsuneraavinash}
\end{rSection}

\begin{rSection}{EDUCATION}
    {\bf University of Moratuwa}                                \hfill {\em September 2017 - Present}
    \\ Undergraduate, Bsc (Computer Science and Engineering)    \hfill { Current SGPA: 4.08/4.2 }
    \\ Dean's List - Semester 1, 2, 3 \par

    {\bf Kalutara Vidyalaya – National School}                  \hfill {\em July 2013 - June 2017}
    \\ GCE Advanced Level Examination                           \hfill { Z-Score: 2.680 }
    \\ All A Passes in Physical Sciences stream (Island rank 76\ss{th}, District Rank 2\ss{nd}) \par

    {\bf Building Cloud Services with the Java Spring Framework}                \hfill {\em Coursera} \\
    {\bf Building Scalable Java Microservices with Spring Boot and Spring Cloud}\hfill {\em Coursera} \\
    % {\bf Google Cloud Platform Fundamentals: Core Infrastructure}               \hfill {\em Coursera} \\
    {\bf RESTful API with HTTP and JavaScript}                                  \hfill {\em Coursera} \\
    {\bf Networking Essentials Course}                                                 \hfill {\em Cisco} \\
    {\bf Introduction to Packet Tracer}                                         \hfill {\em Cisco}
\end{rSection}

%----------------------------------------------------------------------------------------
%	TECHNICAL STRENGTHS SECTION
%----------------------------------------------------------------------------------------

\begin{tSection}{TECHNICAL STRENGTHS}{
        \faCode\ Programming Languages      \ & Java, Python, Dart, JavaScript, Golang, C++, PHP \\
        \faLaptop\ Frameworks \& Technologies\ & Flutter, React, Express, Arduino, OpenCV \\
        \faGlobe\ Web Development           \ & HTML5, CSS, Bootstrap, WordPress \\
        \faDatabase\ Databases              \ & PostgreSql, MySql, Firebase Firestore \\
    }\end{tSection}

%--------------------------------------------------------------------------------
%    Extra-Curricular
%--------------------------------------------------------------------------------

% \begin{rSection}{EXTRA-CURRICULAR}
%     {\bf Teaching Mathematics to rural students | Soyuru Sathkaraya}       \hfill {\em Sept 2017 - Nov 2017}
%     \\"Soyuru Sathkaraya" is an annual volunteer program organized by the Students' Union of University of Moratuwa to improve the mathematics knowledge and abilities of the students in rural schools in Sri Lanka.
% \end{rSection}
% \newpage


%----------------------------------------------------------------------------------------
%	COMPETITION AWARDS
%----------------------------------------------------------------------------------------

\begin{rSection}{COMPETITION AWARDS}{
    {\bf IEEExtreme 12.0 \& 13.0 (IEEE) | Top 100 (global)}                         \hfill {\em October 2018 \& 2019}
    \\A 24-hour international competitive programming competition organized by IEEE and attended by more than 4000 teams. \par

    {\bf ACES Coders v8.0 (University of Peradeniya) | Winners}                     \hfill {\em February 2020}
    \\A 12-hour competitive programming competition organized by the Association of Computer Engineering Students of the Department of Computer Engineering, the University of Peradeniya.\par

    {\bf HackX 2019 (University of Kelaniya) | Winners}                             \hfill {\em September 2019}
    \\An innovative startup challenge conducted by the Management and IT faculty of the University of Kelaniya in which our team developed an Android/iOS mobile app named `Teleport' to facilitate ride-sharing for parcel delivery.\par

    {\bf 14\ss{th} YCS Competition (Ministry of Education \& FITIS) | Winner}       \hfill {\em August 2015}
    \\An island-wide software development competition organized by the Ministry of Education in which my solution was a learning management system designed to teach students who have a negligible knowledge about mathematics.\par

    {\bf XOBot `19 (IESL \& University of Jaffna) | Runners Up}                     \hfill {\em October 2019}
    \\A robotic manipulator design competition conducted as part of the Techno exhibition `19 in which the task was to design a Tic-Tac-Toe playing robotic arm.\par

    {\bf UOJ Coders v1.0 (University of Jaffna) | Winners}                          \hfill {\em March 2019} \\
    {\bf IESL RoboGames '18 (IESL \& University of Moratuwa) | Winners}             \hfill {\em October 2018} \\
    {\bf IDEA Challenge (University of Moratuwa) | Runners Up}                      \hfill {\em October 2018} \\
    {\bf Reakhack 2.0 (University of Kelaniya) | Runners Up}                        \hfill {\em November 2019} \\
    {\bf SLRC `17 (University of Moratuwa) | Second Runners Up}                     \hfill {\em January 2018} \\
    {\bf Hash Code 2019 (Google) | Sri Lankan First Place \& 260\ss{th} globally}   \hfill {\em February 2019} \par

    }\end{rSection}

%--------------------------------------------------------------------------------
%    Honours and Awards (Other)
%--------------------------------------------------------------------------------

\begin{rSection}{HONOURS \& AWARDS (OTHER)}
    {\bf Hiran Chathura Kulasekara Throphy}                                                         \hfill {\em 2016}
    \\Award for best performing student in Physical Science Stream, Kalutara Vidyalaya - National School. \par
    {\bf National School Software Championship (Ministry of Education) | Merit}     \hfill {\em 2015} \\
    {\bf ICT Competition (Ministry of Education) | Western Province 1\ss{st} place}      \hfill {\em 2013}\\
    {\bf Science Competition (Ministry of Education) | Western Province 2\ss{nd} place}   \hfill {\em 2010}\\
    {\bf Social Science Competition (Ministry of Education) | All Island 5\ss{th} place}  \hfill {\em 2010}
\end{rSection}

%--------------------------------------------------------------------------------
%    Projects
%--------------------------------------------------------------------------------

\begin{rSection}{PROJECTS}
    {\bf Theme Provider | Open source plugin}                       \hfill {\em June 2019}
    \\ \lightsubtitle{\url{https://pub.dev/packages/theme\_provider}}\\
    A dependency injection plugin in written \textit{Dart (for Flutter)} to automatically rebuild
    UI on theme changes.
    This package attempts to reduce boilerplate code when adding theme switching functionality to flutter
    while providing theme persistency and support for dark themes out of the box. \par\vspace{5pt}

    {\bf Googong Smart city Community app | Data Visualization applications}                  \hfill {\em April 2020 - Ongoing}
    \\ \lightsubtitle{\url{https://play.google.com/store/apps/details?id=au.com.onewifi.googongIotApp}}\\
    A smart city app which allows users to find local community services,
    receive public announcements of events and upload public amenity
    faults supported by the smart city network in Googong, Australia.
    This was a cross platform mobile application and web application
    developed using \textit{Django and Flutter}.
    \par\vspace{5pt}

    {\bf CB3D Website | 3D Printing Portal}                  \hfill {\em May 2020 - July 2020}
    \\ \lightsubtitle{\url{https://cb3d.circuitbreakerssl.com}}\\
    3D printer website and administration portal with online order management and 3D file preview functionalities.
    \textit{Node.js, Express.js, React and PostgreSql} were used to develop the system.
    \par\vspace{5pt}

    {\bf Rise of the Pharaohs Scavenger hunt app}                   \hfill {\em October 2019 - February 2020}
    \\ \lightsubtitle{\url{https://github.com/kdsuneraavinash/cse-night-app}}\\
    Scavenger-hunt mobile app developed using \textit{Flutter and Firebase} which was created as the invitation
    for the CSE event – Rise of the Pharaohs.  \par\vspace{5pt}

    {\bf Tic-Tac-Toe Playing Robot Arm | Mobile app integrated robot}     \hfill {\em October 2019}
    \\ \lightsubtitle{\url{https://youtu.be/7ki3itajGDc}}\\
    This was a vision-based approach to developing a tic-tac-toe playing robot using \textit{Arduino, Android, and OpenCV}.
    The robot operates via the help of an attached mobile phone which identifies the board configuration
    and calculates the next move by applying the minimax algorithm. \par\vspace{5pt}

    {\bf OIS, Inventory Management System | 5\ss{th} Semester Project}     \hfill {\em February 2020 - June 2020}
    \\ \lightsubtitle{\url{https://github.com/openinventoryorg}}\\
    Automated inventory management system for computer labs with role-based access system.
    The system consisted of a mobile application and a web application.
    The web frontend was developed using \textit{React}, Backend using \textit{Node.js/Express} and
    mobile application using \textit{Flutter}.\par\vspace{5pt}

    {\bf Teleport Mobile App - Cross platform parcel delivery app}                        \hfill {\em August 2019 - September 2019}
    \\
    App developed in order to use sharing economy in transferring goods.
    Built using \textit{flutter}, this app was developed for a proof-of-concept app demonstration for the HackX competition. \par
\end{rSection}

%--------------------------------------------------------------------------------
%    Non-related Refrees
%--------------------------------------------------------------------------------

% \begin{rSection}{Non-Related Refrees}
% \end{rSection}


\end{document}

