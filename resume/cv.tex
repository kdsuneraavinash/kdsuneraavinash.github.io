\documentclass{cv}

\usepackage[left=0.5in,top=0.5in,right=0.5in,bottom=0.5in]{geometry}    % Document margins
\usepackage[fixed]{fontawesome5}
\usepackage{paralist}
% \usepackage[default]{sourcesanspro}
% \usepackage[T1]{fontenc}

\newcommand{\tab}[1]{\hspace{.2667\textwidth}\rlap{#1}}
\newcommand{\itab}[1]{\hspace{0em}\rlap{#1}}

\sidebar{header.png}
\imagename{face.png}
\name{K. D. Sunera Avinash Chandrasiri}                                 % Name
\subtitle{344/1, Moonamalgahawatta, Duwa Temple Road, Kalutara South.}  % Address
\subtitle{(076) 833 6850 \\ \href{mailto:suneraavinash.17@cse.mrt.ac.lk}{suneraavinash.17@cse.mrt.ac.lk}} 
\subtitle{\url{https://kdsuneraavinash.me}} 
\header{true}
% Phone number, email

\begin{document}

\begin{rSection}{PROFILE}
    A third-year computer science undergraduate who has a passion for algorithms and problem-solving.
    I am also interested in web technologies and mobile app development.
    A GNU/Linux Enthusiast. \par
    \faGlobe\ Portfolio: \url{https://kdsuneraavinash.me} \\
    \faGithub\ Github profile: \url{https://github.com/kdsuneraavinash} \\
    \faLinkedin\ LinkedIn profile: \url{https://www.linkedin.com/in/kdsuneraavinash}
\end{rSection}

%----------------------------------------------------------------------------------------
%	TECHNICAL STRENGTHS SECTION
%----------------------------------------------------------------------------------------

\begin{tSection}{TECHNICAL STRENGTHS}{
        \faCode\ Programming Languages      \ & Python, Dart, Java, JavaScript, Golang, C++, PHP \\
        \faLaptop\ Frameworks \& Technologies\ & Flutter, React, Express, Arduino, OpenCV \\
        \faGlobe\ Web Development           \ & HTML5, CSS, Bootstrap, WordPress \\
        \faDatabase\ Databases              \ & PostgreSql, MySql, Firebase Firestore \\
    }\end{tSection}

%----------------------------------------------------------------------------------------
%	EDUCATION SECTION
%----------------------------------------------------------------------------------------

\begin{rSection}{EDUCATION}
    {\bf University of Moratuwa}                                \hfill {\em September 2017 - Present}
    \\ Undergraduate, Bsc (Computer Science and Engineering)    \hfill { Current SGPA: 4.08/4.2 }
    \\ Dean's List - Semester 1, 2, 3 \par

    {\bf Kalutara Vidyalaya – National School}                  \hfill {\em July 2013 - June 2017}
    \\ GCE Advanced Level Examination                           \hfill { Z-Score: 2.680 }
    \\ All A Passes in Physical Sciences stream (Island rank 76\ss{th}, District Rank 2\ss{nd}) \par

    {\bf Courses and Certificates (MOOC)}
    \vspace{-4pt}
    \begin{compactitem}
        \item[$-$]  { Building Cloud Services with the Java Spring Framework}                \hfill {\em Coursera}
        \item[$-$]  { Building Scalable Java Microservices with Spring Boot and Spring Cloud}\hfill {\em Coursera}
        \item[$-$]  { RESTful API with HTTP and JavaScript}                                  \hfill {\em Coursera}
        \item[$-$]  { Networking Essentials Course}                                                 \hfill {\em Cisco}
        \item[$-$]  { Introduction to Packet Tracer}                                         \hfill {\em Cisco}
    \end{compactitem}
\end{rSection}

%--------------------------------------------------------------------------------
%    Extra-Curricular
%--------------------------------------------------------------------------------

% \begin{rSection}{EXTRA-CURRICULAR}
%     {\bf Teaching Mathematics to rural students | Soyuru Sathkaraya}       \hfill {\em Sept 2017 - Nov 2017}
%     \\"Soyuru Sathkaraya" is an annual volunteer program organized by the Students' Union of University of Moratuwa 
%  to improve the mathematics knowledge and abilities of the students in rural schools in Sri Lanka.
% \end{rSection}
% \newpage


%----------------------------------------------------------------------------------------
%	COMPETITION AWARDS
%----------------------------------------------------------------------------------------

\begin{rSection}{COMPETITION AWARDS}{
    {\textbf{IEEExtreme 12.0 (2018) \& 13.0 (2019)} | IEEE}                     \hfill {\em Top 100 (global)}\\
    {\textbf{ACES Coders v8.0 (2020)} | University of Peradeniya}               \hfill {\em Winners}\\
    {\textbf{HackX 2019} | University of Kelaniya}                              \hfill {\em Winners}\\
    {\textbf{UOJ Coders 2019} | University of Jaffna}                           \hfill {\em Winners}\\
    {\textbf{IESL RoboGames '18} | IESL \& University of Moratuwa}              \hfill {\em Winners}\\
    {\textbf{Decrypt Ideathon (2018)} | University of Moratuwa}                 \hfill {\em Winners}\\
    {\textbf{XOBot `19} | IESL \& University of Jaffna}                          \hfill {\em Runners Up}\\
    {\textbf{IESL IDEA Challenge (2018)} | IESL \& University of Moratuwa}      \hfill {\em Runners Up}\\
    {\textbf{Realhack 2.0 (2019)} | University of Kelaniya}                     \hfill {\em Runners Up}\\
    {\textbf{SLRC `17} | University of Moratuwa}                                 \hfill {\em Second Runners Up}\\
    {\textbf{Hash Code 2019} | Google}                                           \hfill {\em Sri Lankan First Place}\par
    }\end{rSection}
\newpage

%--------------------------------------------------------------------------------
%    Honours and Awards (Other)
%--------------------------------------------------------------------------------

\begin{rSection}{HONOURS \& AWARDS (OTHER)}
    {\bf Hiran Chathura Kulasekara Throphy}                                                         \hfill {\em 2016}
    \\Award for best performing student in Physical Science Stream, Kalutara Vidyalaya - National School. \par
    {\bf 14\ss{th} Young Computer Scientist Competition (Ministry of Education) | Gold Medal}     \hfill {\em 2015} \\
    {\bf National Level School Software Championship (Ministry of Education) | Merit}     \hfill {\em 2015} \\
    {\bf ICT Competition (Ministry of Education) | Western Province 1\ss{st} place}      \hfill {\em 2013}\\
    {\bf Science Competition (Ministry of Education) | Western Province 2\ss{nd} place}   \hfill {\em 2010}\\
    {\bf Social Science Competition (Ministry of Education) | All Island 5\ss{th} place}  \hfill {\em 2010}\par
\end{rSection}

%--------------------------------------------------------------------------------
%    Projects
%--------------------------------------------------------------------------------

\begin{rSection}{PROJECTS}
    {\bf Theme Provider | Open source plugin}                       \hfill {\em June 2019}
    \\
    A dependency injection plugin written in \textit{Dart (for Flutter)} to automatically rebuild
    UI on theme changes.
    This package attempts to reduce boilerplate code when adding themepreview switching functionality to flutter
    while providing theme persistency and support for dark themes out of the box.
    \\ \lightsubtitle{ \url{https://pub.dev/packages/theme\_provider}}
    \par\vspace{7pt}

    {\bf Googong Smart city Community app}                  \hfill {\em April 2020 - Ongoing}
    \\
    A smart city app which allows users to find local community services,
    receive public announcements of events and upload public amenity
    faults supported by the smart city network in Googong, Australia.
    This was a cross-platform mobile application and a web application developed
    using \textit{Django and Flutter}.
    \\ \lightsubtitle{ \url{https://play.google.com/store/apps/details?id=au.com.onewifi.googongIotApp}}
    \par\vspace{7pt}

    {\bf CB3D Website | 3D Printing Portal}                  \hfill {\em May 2020 - August 2020}
    \\
    3D printer website and administration portal with online order management and 3D file preview functionalities.
    \textit{Node.js, Express.js, React and PostgreSql} were used to develop the system.
    \\ \lightsubtitle{ \url{https://cb3d.circuitbreakerssl.com}}
    \par\vspace{7pt}

    {\bf Tic-Tac-Toe Playing Robot Arm | Mobile app integrated robot}     \hfill {\em October 2019}
    \\
    This was a vision-based approach to developing a tic-tac-toe playing robot using \textit{Arduino, Android, and OpenCV}.
    The robot operates via the help of an attached mobile phone which identifies the board configuration
    and calculates the next move by applying the minimax algorithm.
    \\ \lightsubtitle{ \url{https://youtu.be/7ki3itajGDc}}
    \par\vspace{7pt}

    {\bf Rise of the Pharaohs Scavenger hunt app}                   \hfill {\em October 2019 - February 2020}
    \\
    Scavenger-hunt mobile app developed using \textit{Flutter and Firebase} which was created as the invitation
    for the CSE event – Rise of the Pharaohs.
    \\ \lightsubtitle{ \url{https://github.com/kdsuneraavinash/cse-night-app}}
    \par\vspace{7pt}

    {\bf Java based RPAL Interpreter}     \hfill {\em June 2020}
    \\
    Interpreter implemented in Java which evaluates abstract syntax trees according
    to the RPAL language grammar specification. Developed as part of the CS3152 Programming Languages module.
    \\ \lightsubtitle{ \url{https://github.com/kdsuneraavinash/rpal-ast-interpreter}}
    \par\vspace{7pt}

    {\bf Open Inventory System | 5\ss{th} Semester Project}     \hfill {\em February 2020 - June 2020}
    \\
    Automated inventory management system for computer labs with a role-based access system.
    The system consisted of a mobile application and a web application.
    \\ \lightsubtitle{ \url{https://github.com/openinventoryorg}}
    \par

    % \vspace{7pt}

    % {\bf GI Tract Image Classifier using CNN | 4\ss{th} Semester Project}                        \hfill {\em August 2019 - December 2019}
    % \\
    % Project on classifying anomalies in the gastrointestinal tract through endoscopic imagery
    % using transfer learning with pretrained CNN models such as inception, resnet34 and VGGNet.
    % This was done as a part of the module CS2212 - Programming Challenge II.
    % \\ \lightsubtitle{ \url{https://github.com/kdsuneraavinash/kvasir}}
    % \par

    % {\bf Teleport Mobile App - Cross platform parcel delivery app}                        \hfill {\em August 2019 - September 2019}
    % \\
    % App developed in order to use sharing economy in transferring goods.
    % Built using \textit{flutter}, this app was developed for a proof-of-concept app demonstration for the HackX competition. \par
\end{rSection}

%--------------------------------------------------------------------------------
%    Non-related Refrees
%--------------------------------------------------------------------------------

\begin{rSection}{NON-RELATED REFEREES}
    {\bf Dr. Dulani Meedeniya | Senior Lecturer}
    \\ \lightsubtitle{ \href{mailto:dulanim@cse.mrt.ac.lk}{dulanim@cse.mrt.ac.lk}}
    \\ Department of Computer Science and Engineering, \\
    University of Moratuwa,
    Sri Lanka.\par
\end{rSection}


\end{document}

